\chapter[Implementation]{Implementation \protect{\estimatedpagecount{20}}}
\epigraph{What runs, the code or the comments?}{Brian Kernighan}

\todo[inline]{Mention the idea of using idiomatic \CS, so that legacy projects can be supported}
\todo[inline]{Show the pathway on how this developed...}

\section{Types of Holes}
\subsection{Todo-Comment}
\todo[inline]{Explain the necessity of Todo-Comments, how they can look like and why it makes sense to support them in their purest way}

\subsection{Hole-Comment}
\todo[inline]{Explain the difference between Hole-Comments and Todo-Comments and what lifts a Todo-Comment in the Hole-Comment state}

\subsection{Missing Implementation}
\todo[inline]{Reference the opening quote of this chapter}
% - https://softwareengineering.stackexchange.com/a/323503
\todo[inline]{Explain the concept of stub-generation by the IDE, why it makes sense, but why it is scary that those NotImplementedExceptions are only discovered at runtime}
\todo[inline,color=blue!40]{Maybe do a small excursion into the land of checked exceptions and the difference between \CS and Java}

\subsection{Side-Effect}
\todo[inline]{Show the missing piece of runtime holes and which power they provide.}
\todo[inline,color=blue!40]{Maybe already dive into the idea of abstract I/O and that getters and setters are enough to model side-effects}


\section{Architecture \protect{\estimatedpagecount{2}}}
\todo[inline]{Explain the overall architecture of the combination of a library, the analyzers and the VS Plugin}
\todo[inline]{Explain the difficulties in building a DSL in \CS, no functions, only static methods on classes}

\subsection{Architecture Overview}
\todo[inline,color=blue!40]{Maybe include a C4 diagram of the packages}

\subsection{Bridging static Contexts and DI}
\todo[inline]{Explain the difficulties of bridging the static context and DI}

\subsection{Supporting Logging}
\todo[inline]{Dive into .NET's mess of ILogger, ILogger<T> and ILoggerFactory and their usage in libraries}
\todo[inline]{Explain how this could be solved using StashBox}

\subsection{Language Independence}
\todo[inline]{Explain the modularity and how the \CS-solution might be transferred to other languages (e.g. Python, Java, TS)}
\todo[inline]{mention what is needed for other languages to adopt holes}

\section{Providing IDE Support}
\todo[inline]{Give a quick introduction into Roslyn Analyzers}
\todo[inline]{Explain how Roslyn Analyzers offer IDE-independent support}

\subsection[Analyzers]{Analyzers \protect{\estimatedpagecount{3}}}
\todo[inline]{Explain the implemented Analyzers}
\todo[inline]{Explain why it makes sense that debug and release mode are handled separately and how this was accomplished}
\todo[inline]{Connect this to TreatWarningsAsErrors}
\todo[inline]{Explain the difference to simple linters}

\subsection{Code Fixes}
\todo[inline]{Show the power and usability improvements that Code Fixes in combination with Analyzers provide}

\section{Bridging Compile- and Runtime}
\todo[inline]{Explain why it is necessary to bridge between compile- and runtime}

\subsection{Stacktraces}
\todo[inline]{Dig into the concept of Stacktraces and why they might be (but in reality can't) be used to get information about the running code}

\subsection{Compiler Generated Attributes}
\todo[inline]{Explain the usage of .NET's compiler generated attributes}
\todo[inline]{Also mention the downside of having to specify generics when strings are used}

\subsection{Source Generators}
\todo[inline]{Briefly explain \CS's version of source generators vs. T4 templates}
\todo[inline]{Explain how compile-time information can be lifted into runtime}

\section{Extensibility}
\todo[inline]{Focus on the importance of Extensibility for such an abstract concept/library}

\subsection{Reporting}
\todo[inline]{Explain how custom reporters can be utilized}

\subsection{Mocking}
\todo[inline]{Explain the pathway to prototype-mocking as well as why custom packages aren't necessary anymore}
\todo[inline]{Once again mention the concept of abstract I/O and how powerful this is}
\todo[inline]{Transition to the Sidecar Application}

\subsection{Sidecar Application}
\todo[inline]{Briefly explain that the Sidecar Application is no fundamental part of Holey, but it can be created solely on Holey's extensibility options}

\subsubsection{Communication}
\todo[inline]{Explain the abstracted communication channels and that they are not tied to anything else}

\subsubsection{Dynamic Form Generation}
\todo[inline]{Quickly dive into dynamic form generation and show all the code (because it is just one line per application)}
\todo[inline,color=blue!40]{Maybe also mention that the JSON Schema adapter can be switched to a custom one}
