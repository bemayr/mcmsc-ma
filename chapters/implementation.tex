\chapter[Implementation]{Implementation \protect{\estimatedpagecount{20}}}
\epigraph{What runs, the code or the comments?}{Brian Kernighan}
- using idiomatic C\#, so that legacy projects can be supported + path to this decision
\section{Types of Holes}
\subsection{Todo-Comment}
\subsection{Hole-Comment}
\subsection{Missing Implementation}
- https://softwareengineering.stackexchange.com/a/323503
\subsection{Side-Effect}
\section{Architecture (2)}
\subsection{Architecture Overview}
- maybe include a C4-Diagram of the packages
\subsection{Bridging static Contexts and DI}
\subsection{Supporting Logging}
\subsection{Language Independence}
\todo{mention what is needed for other languages to adopt holes}
\section{Providing IDE Support}
\subsection{Analyzers (3)}
- how analyzers work, and in-depth description of implemented analyzers
- debug vs. release
- TreatWarningsAsErrors
- vs linting
\subsection{Code Fixes}
- testing Code Fixes + Analyzers
\section{Bridging Compile- and Runtime}
\subsection{Stacktraces}
\subsection{Compiler Generated Attributes}
- mention downside of having to specify generics when using strings
\subsection{Source Generators}
- source generators
- run-time vs. compile-time + stacktrace vs. [Attributes]
- lifting compile time information to runtime information
\section{static contexts + DI}
\section{Supporting Logging + Reporting}
\section{Extensibility}
\subsection{Reporting}
\subsection{Mocking}
\todo{Abstract I/O}
\todo{Reference the Sidecar Application}
\subsection{Sidecar Application}
\subsubsection{Communication}
\subsubsection{Dynamic Form Generation}