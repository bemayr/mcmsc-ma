\chapter[Introduction]{Introduction \protect{\estimatedpagecount{16}}}
\epigraph{How do we get people to understand programming?}{Bret Victor}

The idea of learnable programming \cite{victor_inventing_2012} goes back to the early days of Smalltalk \cite{kay_early_1993}.
In his essay \emph{The Early History of Smalltalk}, Alan Kay looks back at the events that led to the creation of Smalltalk.
More importantly, he gives insight about the context of why Smalltalk turned out how it did.
Smalltalk was intended to be a learning environment for children, side-effects like its strong object orientation were merely artifacts of this main idea \cite{kay_early_1993}.
Objects, message passing and compositionality allowed people (not only children) using Smalltalk, to align their ways of thinking with the way their programs were structured.
Applying the conversational lens of programming environments (see Section~\ref{sec:conversational-lens}) to Smalltalk one could argue that speaking the same language results in a shared mental model between the program and the programmer.

Programming environments that foster this style of communication are usually called live programming\cite{aguiar_live_2019, church_liveness_2010} or interactive programming \cite{czaplicki_interactive_2013, mccabe_towards_2023}.
Usually, such systems possess the following properties \cite{burg_1st_2013}:
\begin{description}
    \item[Live] They give the programmer immediate feedback, while the program is edited. This feedback can target its output, structure or both.
    \item[Structured] Environments which are aware of a program's structure, understand and preserve this structure. They operate at the level of structure, not on raw text.
    \item[Tangible] Normally, program execution happens behind the scenes and provides little to no capabilities for live inspection. Tangible programming makes the execution transparent, tangible and explorable.
    \item[Concrete] It is way easier for people to start with concrete examples and generalize later on.
\end{description}

\todo[inline]{Transition to the arrow and bow vs. firehose analogy from Chris Hancock (explained by \cite{aguiar_live_2019}) and explain the difference between aiming. Delta-Time between action and feedback, vs. consistent re-aiming}.
\todo[inline]{This feedback-loop metaphor can be applied to a specific developer, a team or even with customer-interaction.}
\todo[inline,color=blue!40]{Maybe cross-reference to Psychology of Programming and explain that faster feedback fosters happiness.}

\begin{figure}[h]
\centering
\includesvg[width=0.75\textwidth]{images/arrow-bow}
\caption{When trying to hit a target with bow and arrow one can only re-aim based on the previous shot. Feedback is provided only after the arrow reaches the target. Image source~\cite{aguiar_live_2019}.}
\label{fig:visual-studio-error-list}
\end{figure}

\begin{figure}[h]
\centering
\includesvg[width=0.6\textwidth]{images/firehose}
\caption{Aiming with a hose is much easier. Continuous feedback allows the person using the hose to constantly re-adapt to the target, without having to stop aiming at all. Image source~\cite{aguiar_live_2019}.}
\label{fig:visual-studio-error-list}
\end{figure}

\todo[inline]{Explain that industry is slowly adopting these ideas (based on [Kubelka.etal:2018]).}
\todo[inline]{Argue that there is much better we can do, and this is the content of this thesis. Then transition to the context of the thesis.}

% Liveness in practice. Several frameworks and tools nowadays add ways to integrate immediate feedback, particularly on the web. The Chrome web development tools [ 4] allow changes to web pages and JavaScript code without reloading a webpage. Modern, popular JavaScript frameworks including Facebook’s React.js support Live Programming in live code editors [3]. \cite{kubelka_road_2018}
% Mainstream languages also take steps in this direction: Java supports a limited replacement of application parts while it is running through the Java Platform Debugger Architecture [24 ]. Apple’s new language Swift supports Live Programming with interactive playgrounds [1 ]. Finally, Microsoft’s Visual Studio 2015 supports interactive programming for \CS and F\# with REPLs [25 , 26 ]. Microsoft also recently acquired the CodeConnect startup, which developed the Alive extension for Visual Studio [2].



\section[Context]{Context \protect{\estimatedpagecount{3.5}}}
\todo[inline]{In the introduction I did not differentiate between programming and software development, but it is a huge differentiation.}
\todo[inline]{Explain the differences between psychology of programming and psychology of software development and what it has to do with requirements to code and idea to code.}

\subsection[Agile Movement]{Agile Movement \protect{\estimatedpagecount{1.5}}}
\todo[inline]{Give an overview on the development of agile ideas and how it has changed the tasks of software developers.}
\todo[inline]{Use the arrow/firehose analogy to mentally visualize the faster iteration cycles leading to tighter feedback loops}
\todo[inline]{Software has to be developed faster, the requirements are defined more dynamically - mention theory vs. reality}
\todo[inline]{Bring the concept of tactical programming vs. strategic programming into play (based on Osterhout as described in my Psychology BA)}
\todo[inline]{Transition to the next chapter by highlighting the importance of prototypes in agile development.}
\todo[inline,color=blue!40]{Maybe look into Rapid Application Development and its approach on Prototyping}

\subsection[Prototyping]{Prototyping \protect{\estimatedpagecount{1.5}}}
\todo[inline]{Start with the etymology of the word prototype, based on greek proto + typos.}
\todo[inline]{Explain how prototyping works in reality and what evolutionary prototyping is. (with the figure from my AmA talk)}
\todo[inline]{Mention that prototypes get released to production (Tesla, Spotify, Apple) and that customers do the testing.}
\todo[inline]{Show this contradiction with the etymological approach.}
\todo[inline,color=blue!40]{Maybe bring in a little bit of capitalism criticism in the form of market fit vs. high-quality products, just show the contradiction.}
\todo[inline]{Transition to wizard of oz prototyping and how this is actually prototyping, but show its complexity and the difference between the prototype and the actual product.}

\subsection[Interactive Programming]{Interactive Programming \protect{\estimatedpagecount{.5}}}
\todo[inline]{Bridge the gap between prototyping and interactive/live programming as described in the introduction. Mention how this might align with agile methodologies, the psychology of programming and customer-centered approaches.}



\section[Problem Statement]{Problem Statement \protect{\estimatedpagecount{6}}}
\todo[inline]{Based on the context, explain the contradictions once more and re-phrase them as the problem statement of this thesis.}

\subsection[Programming as Complex Problem Solving]{Programming as Complex Problem Solving \protect{\estimatedpagecount{4}}}
\todo[inline]{Use material from the Psychology BA to show similarities between research on complex problem solving and programming.}

\subsubsection[Top-Down vs. Bottom Up]{Top-Down vs. Bottom Up \protect{\estimatedpagecount{1.5}}}
\todo[inline]{Explain top down and bottom up processes and how they problem solving applies to it via functional (de-)composition.}
\todo[inline]{Create figures that look like fractal pyramides, one approaching from top, one approaching from bottom.}
\todo[inline]{"To use the bottom up method you need to be able to efficiently determine what the "bottom" is, which usually means you need a heavily constrained problem space."}
\todo[inline,color=green!40]{relevant literature: \cite{kung_comparing_2013}}
% \url{https://en.wikipedia.org/wiki/Bottom%E2%80%93up_and_top%E2%80%93down_design}
% \url{https://www.geeksforgeeks.org/difference-between-bottom-up-model-and-top-down-model/}

\subsubsection[Programming as Theory Building]{Programming as Theory Building \protect{\estimatedpagecount{1.5}}}
\label{sec:programming-as-theory-building}
\todo[inline]{Deep dive into Peter Naur's perspective on programming and why theory building explains why software is hard to maintain.}
\todo[inline]{Maybe bring the example of the Compiler for showing how communication is such an important aspect of software development.}
\todo[inline]{\textbf{Basically sum up his paper in one and a half pages, cause it is so fundamental to this research.}}
\todo[inline,color=green!40]{relevant literature: \cite{naur_programming_1985}}

% In a very simplistic view, programming can be seen as the act of converting requirements to source code.
% With this task being carried out by people, one could look at psychology's research field of complex problem-solving and apply it to programming.
% There is not a lot of active research being conducted in this area, but Weinberg \cite{Weinberg:1971} and Naur \cite{Naur:1985} did some fundamental research combining psychology's complex problem solving and the act of programming.
% In his essay \emph{Programming as Theory Building} \citet{Naur:1985} hypothesized that programs are more than just their textual artifacts.
% Instead, individuals, as well as teams create a shared understanding of the problem they have to solve.
% Programming is about creating this understanding, hence improving programming is about improving the creation of this understanding, thus improving communication.

% Communication does not only happen in teams.
% Someone, developing something that can not be accomplished in a few minutes, needs to communicate with oneself as well as the computer.
% Decomposing requirements, keeping track of them, discovering bugs, and transforming code changes are just some of the tasks that might need some form of communication.
% Quite a lot of this communication happens via code comments \cite{Ying.etal:2005}.
% Comments are not only used for documentation purposes, but as well to make notes of things that have to be changed or considered.
% In contrast to the parts of source code that are actually run, comments are informal and not bound to any syntax rules, making them hard to parse and interpret for machines.
% This leads to comments being forgotten quite frequently \cite{Nie.etal:2019}.
% Preventing their forgetting and actually making use of them for the act of programming is the main motivation for this research.

\subsubsection[Conversational Lens regarding Complex Problem Solving in Development]{Conversational Lens regarding Complex Problem Solving in Development \protect{\estimatedpagecount{1}}}
\label{sec:conversational-lens}
\todo[inline]{Sum up Alan's ideas regarding looking at programming through the conversational lens and how the idea of communication can be applied to IDEs.}
\todo[inline]{Also put this conversational lens idea into perspective regarding conversations between people, like already done in the MuC paper and described in \ref{sec:programming-as-theory-building}.}
\todo[inline,color=green!40]{relevant literature: \cite{mccabe_towards_2023}}

\subsection[Mental Load]{Mental Load \protect{\estimatedpagecount{2.5}}}
\todo[inline]{Check what "Code that fits in your Head" can attribute to this section}
\todo[inline]{Dive a bit into the psychology of programming and its research on \emph{working memory capacity}}
\todo[inline]{Present the research about Post-Its}
\todo[inline,color=blue!40]{Maybe look into the psychological implications like stress, anxiety and other disorders}
\todo[inline,color=green!40]{relevant literature: \cite{digiano_learning_nodate, seemann_code_2021}}



\section[Hypothesis]{Hypothesis \protect{\estimatedpagecount{4}}}
\todo[inline]{Transition to the hypothesis section of this paper, because of it being a Proof of Concept}

\subsection[About Todo-Comments]{About Todo-Comments \protect{\estimatedpagecount{2.5}}}
\todo[inline]{Start with an introduction on scientific results regarding Todo comments}
\todo[inline]{Do a short qualitative analysis regarding the various stackexchange posts on Todo-Comments}
\todo[inline,color=blue!40]{If I manage to find, I intend to include some statistics on Todo-Comments}
% - \url{https://www.petermorlion.com/the-lifetime-of-todo-comments-the-results/}
% - Tickgit
\todo[inline,color=blue!40]{Maybe try to group and order these into separate categories/quantitites, or report them by stackexchange question/discussion}
% - https://softwareengineering.stackexchange.com/questions/175719/can-notes-to-dos-in-code-comments-sent-to-code-reviews-result-in-an-effective-re
% - \url{https://softwareengineering.stackexchange.com/questions/323498/why-is-having-a-notimplementedexception-a-good-thing#comment687389_323498}
% - https://softwareengineering.stackexchange.com/questions/125320/do-todo-comments-make-sense
% - https://stackoverflow.com/questions/1989177/how-to-manage-todo-programming-stuff
% - https://stackoverflow.com/questions/16913055/how-can-i-mark-to-do-comments-in-xcode
% - https://stackoverflow.com/questions/335378/how-do-you-flag-code-so-that-you-can-come-back-later-and-work-on-it
\todo[inline,color=green!40]{relevant literature: \cite{nie_natural_2018, ying_source_2005, nie_framework_2019, sridhara_automatically_2016, storey_todo_2008, storey_how_2009}}

\subsection[Introducing Hole-Driven Development]{Introducing Hole-Driven Development \protect{\estimatedpagecount{1.5}}}
\todo[inline]{Start with the \emph{Fill in the Blank Exercise} analogy, then continue with an example}
\todo[inline]{Quickly show the Idris example, but mainly reference the related work section}

\section[Research Question and Methodology]{Research Question and Methodology \protect{\estimatedpagecount{1.5}}}
\todo[inline]{Lay out the process of approaching this thesis}
\todo[inline]{It mainly consists of deducing the requirements based on literature (reference the according chapter in related work), and validating the results according to the already supported concepts of Hole-Driven Development}
\todo[inline]{\emph{This work represents a proof of concept for applying the concept of hole-driven development to general-purpose programming languages, in this case \CS.}}
\todo[inline]{\emph{Which aspects of existing Hole Implementations can be transferred to general-purpose (mainstream) programming languages? With the requirements being laid in Sec 2.4 out after analyzing those in Sec 2.}}

\section[Structure of the Thesis]{Structure of the Thesis \protect{\estimatedpagecount{.5}}}
\todo[inline]{Quickly describe the structure of the thesis.}