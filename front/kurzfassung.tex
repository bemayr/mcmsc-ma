\chapter{Kurzfassung}

\begin{german} %switch to German language rules
    Live-Programming-Umgebungen ermöglichen das interaktive Explorieren von Software, wodurch das Konstruieren mentaler Modelle erleichtert wird und eben diese Software besser verstanden werden kann.
    Deren Eigenschaft von besonders kurzen Feedback-Schleifen ermöglicht das schnelle Testen neuer Ideen, ohne den Fokus zu verlieren.
    Schnelleres und akkurateres Feedback führte außerdem zur Entwicklung agiler Vorgehensmethoden, welche durch den Fokus auf Feedback, Individuen und Kommunikation gekennzeichnet sind.
    Auch diese weniger prozessorientierten Vorgehensmodelle profitieren von Werkzeugen, besonders wenn diese der Kommunikation dienen.
    Betrachtet man Programmieren als Spezialisierung des Forschungsgebietes des komplexen Problemlösens, sind Werkzeuge insbesondere dann hilfreich, wenn diese es schaffen die kognitive Belastung der Personen zu reduzieren.
    Prototypen zeigen, wie sich Software verhält bzw. verhalten könnte und sind somit dazu geeignet, ein gemeinsames Verständnis über die Anforderungen von Software zu erlangen.
    Viele der am häufigsten genutzten Programmiersprachen sind allerdings nicht darauf ausgelegt, mithilfe von kurzen Feedback-Schleifen interaktive Prototypen zu erstellen, deren Programmcode so weiterentwickelt werden kann, dass die Qualität der Software Industriestandards entspricht.

    Durch das Betrachten des Softwareentwicklungsprozesses als Prozess der Kommunikation identifizierten wir bestehende Lösungen, welche dabei helfen kognitive Belastungen zu minimieren und kurze Feedback-Schleifen ermöglichen.
    Basierend auf Todo-Kommentaren, dem Konzept des Hole-Driven Developments und kreativen Lösungen zum Markieren von unfertigen Programmteilen identifizierten wir 13 Eigenschaften von Programmteilen, die als unfertig gekennzeichnet sind.
    Basierend auf diesen Eigenschaften entwickelten wir \emph{Holey}, einen Ansatz zur Verwendung von Hole-Driven Development in der weit verbreiteten Programmiersprache \CS.
    Wir zeigen, dass es möglich ist, unter Verwendung von existierenden Sprachkonstrukten und einer erweiterbaren Bibliothek, Hole-Driven Development in\ \CS zu verwenden.
    Basierend auf dieser Machbarkeitsstudie kann untersucht werden, inwieweit das Entwickeln von lediglich teils fertigen, jedoch ausführ- und weiterentwickelbaren Anwendungen Vorteile bietet.
\end{german}