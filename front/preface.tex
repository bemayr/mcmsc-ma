\chapter{Preface}
Finishing this thesis took longer than anticipated.
There is a bit of backstory to this.
After completing my first bachelor's in software engineering, I started another bachelor's in psychology, which sparked my interest in combining those seemingly contradictory research areas.
I started researching the topic of holes in programming when I wrote my bachelor's thesis in psychology about the possibilities of combining holes with Statecharts.

Shortly after this, I joined the fantastic people at StateML, where we are researching a language for modeling event-driven behavior in our spare time.
Thank you, Chris, John, and Owen, for the countless hours of discussions about the history of software, non-conventional ideas of approaching software, and what we can learn from this.
We already integrated some ideas around holes into StateML; therefore, I initially started working on this thesis to show how to apply Hole-Driven Development to Statecharts.
However, your immense focus on composability reshaped my view of holes.
As such, this thesis is about creating a framework for composable holes instead of applying it to Statecharts directly.
But this idea is not forgotten; it was simply turned into a hole to be filled later on.

Parts of the core idea of this thesis have already been presented at Mensch und Computer (MuC) 23 and published in \cite{mayr_replacing_2023}.
Thank you, Marc, for motivating me to submit at MuC, providing feedback about my ideas, and for a great time in Switzerland.

Prior to presenting at MuC, I attended the Psychology of Programming Interest Group's (PPIG) 34th annual workshop in Lund, Sweden.
Thank you, Clayton, Luke, Leah, Anton, and Görel, for your input, and Alan, for your exceptional defense and the conversational perspective regarding programming.

Speaking of input, many ideas are born, enhanced, rejected, and validated at this inspiring monthly round table with Alex, Dominik, Georg, Sebastian, and Stefan.
You played a big part in why I returned to software development and still stick around; thanks for all the inspiring conversations about everything.

The people who have probably been most affected by me writing this thesis are my family, flatmates, and friends.
Thank you for all your support and sorry for the times I have been unresponsive, especially Mom, Tinka, Mona, Jakob, Rosa, Daniel, David, and Marlene!

I feel really privileged that my dad had the greatest influence on this thesis.
He challenges my beliefs, lets me challenge his, and can give me hints about decades of experience in software engineering.
He is also the best proofreader, knowing so many details about grammar, scientific writing, and the topics of not only my technical writings - thank you, Dad!

This thesis would not have been possible without my supervisor Stephan Selinger.
Thank you for your input, patience, trust, and flexibility, and for supervising a thesis unrelated to machine learning or large language models in 2023.
I can still not imagine doing an Ironman after completing the 100km trail run, but I've got the feeling that writing this thesis might have brought me a little bit closer.

\vspace{6ex}
\noindent
B.\ Mayr